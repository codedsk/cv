%%%%%%%%%%%%%%%%%%%%%%%%%%%%%%%%%%%%%%%%%%%%%%%%%%%%%%%%%%%%%%%%%%%%%%%%
%%%%%%%%%%%%%%%%%%%%%% Simple LaTeX CV Template %%%%%%%%%%%%%%%%%%%%%%%%
%%%%%%%%%%%%%%%%%%%%%%%%%%%%%%%%%%%%%%%%%%%%%%%%%%%%%%%%%%%%%%%%%%%%%%%%

%%%%%%%%%%%%%%%%%%%%%%%%%%%%%%%%%%%%%%%%%%%%%%%%%%%%%%%%%%%%%%%%%%%%%%%%
%% NOTE: If you find that it says                                     %%
%%                                                                    %%
%%                           1 of ??                                  %%
%%                                                                    %%
%% at the bottom of your first page, this means that the AUX file     %%
%% was not available when you ran LaTeX on this source. Simply RERUN  %%
%% LaTeX to get the ``??'' replaced with the number of the last page  %%
%% of the document. The AUX file will be generated on the first run   %%
%% of LaTeX and used on the second run to fill in all of the          %%
%% references.                                                        %%
%%%%%%%%%%%%%%%%%%%%%%%%%%%%%%%%%%%%%%%%%%%%%%%%%%%%%%%%%%%%%%%%%%%%%%%%

%%%%%%%%%%%%%%%%%%%%%%%%%%%% Document Setup %%%%%%%%%%%%%%%%%%%%%%%%%%%%

% Don't like 10pt? Try 11pt or 12pt
\documentclass[10pt]{article}
\RequirePackage[T1]{fontenc}

% The automated optical recognition software used to digitize resume
% information works best with fonts that do not have serifs. This
% command uses a sans serif font throughout. Uncomment both lines (or at
% least the second) to restore a Roman font (i.e., a font with serifs).
\usepackage{times}
\renewcommand{\familydefault}{\sfdefault}

% The OCR software also has a hard time with italics. These commands get
% rid of the two common ways to italicize text in LaTeX. Get rid of them
% to turn italics back on.
\renewcommand\emph[1]{#1}
\renewcommand\textit[1]{\underline{\smash{#1}}}

% This is a helpful package that puts math inside length specifications
\usepackage{calc}

% This package helps LaTeX auto-hyphenate hyphenated words if you use
% special hyphens. For example, bio\-/mimicry will properly hyphenate
% ``mimicry'' if necessary.
\usepackage[shortcuts]{extdash}

% Layout: Puts the section titles on left side of page
\reversemarginpar

%
%         PAPER SIZE, PAGE NUMBER, AND DOCUMENT LAYOUT NOTES:
%
% The next \usepackage line changes the layout for CV style section
% headings as marginal notes. It also sets up the paper size as either
% letter or A4. By default, letter was used. If A4 paper is desired,
% comment out the letterpaper lines and uncomment the a4paper lines.
%
% As you can see, the margin widths and section title widths can be
% easily adjusted.
%
% ALSO: Notice that the includefoot option can be commented OUT in order
% to put the PAGE NUMBER *IN* the bottom margin. This will make the
% effective text area larger.
%
% IF YOU WISH TO REMOVE THE ``of LASTPAGE'' next to each page number,
% see the note about the +LP and -LP lines below. Comment out the +LP
% and uncomment the -LP.
%
% IF YOU WISH TO REMOVE PAGE NUMBERS, be sure that the includefoot line
% is uncommented and ALSO uncomment the \pagestyle{empty} a few lines
% below.
%

%% Use these lines for letter-sized paper
\usepackage[paper=letterpaper,
            %includefoot, % Uncomment to put page number above margin
            marginparwidth=1.2in,     % Length of section titles
            marginparsep=.05in,       % Space between titles and text
            margin=1in,               % 1 inch margins
            includemp]{geometry}

%% Use these lines for A4-sized paper
%\usepackage[paper=a4paper,
%            %includefoot, % Uncomment to put page number above margin
%            marginparwidth=30.5mm,    % Length of section titles
%            marginparsep=1.5mm,       % Space between titles and text
%            margin=25mm,              % 25mm margins
%            includemp]{geometry}

%% More layout: Get rid of indenting throughout entire document
\setlength{\parindent}{0in}

% Provides special list environments and macros to create new ones
\usepackage[shortlabels]{enumitem}

% Simpler bibsections for CV sections
% (thanks to natbib for inspiration)
%
% * For lists of references with hanging indents and no numbers:
%
%   \begin{bibsection}
%       \item ...
%   \end{bibsection}
%
% * For numbered lists of references (with hanging indents):
%
%   \begin{bibenum}
%       \item ...
%   \end{bibenum}
%
%   Note that bibenum numbers continuously throughout. To reset the
%   counter, use
%
%   \restartlist{bibenum}
%
%   at the place where you want the numbering to reset.

\makeatletter
\newlength{\bibhang}
\setlength{\bibhang}{1em}
\newlength{\bibsep}
 {\@listi \global\bibsep\itemsep \global\advance\bibsep by\parsep}
\newlist{bibsection}{itemize}{3}
\setlist[bibsection]{label=,leftmargin=\bibhang,%
        itemindent=-\bibhang,
        itemsep=\bibsep,parsep=\z@,partopsep=0pt,
        topsep=0pt}
\newlist{bibenum}{enumerate}{3}
\setlist[bibenum]{label=[\arabic*],resume,leftmargin={\bibhang+\widthof{[999]}},%
        itemindent=-\bibhang,
        itemsep=\bibsep,parsep=\z@,partopsep=0pt,
        topsep=0pt}
\let\oldendbibenum\endbibenum
\def\endbibenum{\oldendbibenum\vspace{-.6\baselineskip}}
\let\oldendbibsection\endbibsection
\def\endbibsection{\oldendbibsection\vspace{-.6\baselineskip}}
\makeatother

%%% Setup header and footer (with page number and possible last page)
%
% The first block sets up pages 2--end
% The second block sets up page 1 formatting
%
%%%
%
% NOTE: comment the +LP lines and uncomment the -LP lines to have page
%       numbers without the ``of ##'' last page reference)
%
% NOTE: uncomment the \pagestyle{empty} line to get rid of all page
%       numbers on pages 2--end. To get rid of page numbers on page 1,
%       comment out the \thispagestyle{plain} line on the first page
%       below.
%       (also make sure includefoot is commented out above)
%
\usepackage{fancyhdr,lastpage}
\pagestyle{fancy}
%\pagestyle{empty}      % Uncomment this to get rid of page numbers
\fancyhf{}\renewcommand{\headrulewidth}{0pt}
\fancyfootoffset{\marginparsep+\marginparwidth}
\newlength{\footpageshift}
\setlength{\footpageshift}
          {0.5\textwidth+0.5\marginparsep+0.5\marginparwidth-2in}

%%%% PAGES 2--9 NUMBERING:
%% These two lines put page number in upper-right corner of pages 2--end
\rhead{Kearney, p.~\arabic{page} of \protect\pageref*{LastPage}}   % +LP
%\rhead{Pavlic, p.~\arabic{page}}                                 % -LP

%% These lines put page number in bottom (center) of pages 2--end
%\lfoot{\hspace{\footpageshift}%
%       \parbox{4in}{\, \hfill %
%                    \arabic{page} of \protect\pageref*{LastPage} % +LP
%%                    \arabic{page}                               % -LP
%                    \hfill \,}}
%%%% END PAGE 2--9 NUMBERING

%%%% PAGE 1 NUMBERING:
\makeatletter
\let\oldps@plain\ps@plain
\renewcommand{\ps@plain}{\oldps@plain%
\renewcommand{\@evenfoot}{\hspace*{-\footpageshift}\hfil %
    p.~\arabic{page} of \protect\pageref*{LastPage} % +LP
%    p.~\arabic{page}                               % -LP
    \hfil}%
\renewcommand{\@oddfoot}{\@evenfoot}}
\makeatother
%%%% END PAGE 1 NUMBERING

% Finally, give us PDF bookmarks and colored links
%
% NOTE: Some OCR software might be negatively affected by hyperlinks. So
%       most employers recommend the draft option here. Alternatively,
%       making all links black (as opposed to darkblue) should hopefully
%       prevent problems with most OCR.
%
% (to enable hyperlinks and bookmarks, comment out ``draft'' line;
%  to disable hyperlinks and bookmarks, uncomment ``draft'' line)
\usepackage{color,hyperref}
\definecolor{darkblue}{rgb}{0.0,0.0,0.3}
\hypersetup{breaklinks,colorlinks,
            linkcolor=black,urlcolor=black,
            anchorcolor=black,citecolor=black,
            %linkcolor=darkblue,urlcolor=darkblue,
            %anchorcolor=darkblue,citecolor=darkblue,
            %draft
            }

%%%%%%%%%%%%%%%%%%%%%%%% End Document Setup %%%%%%%%%%%%%%%%%%%%%%%%%%%%


%%%%%%%%%%%%%%%%%%%%%%%%%%% Helper Commands %%%%%%%%%%%%%%%%%%%%%%%%%%%%

%%% HEADING AT TOP OF CURRICULUM VITAE

% The title (name) with a horizontal rule under it
% (optional argument typesets an object right-justified across from name
%  as well)
%
% Usage: \makeheading{name}
%        OR
%        \makeheading[right_object]{name}
%
% Place at top of document. It should be the first thing.
% If ``right_object'' is provided in the square-braced optional
% argument, it will be right justified on the same line as ``name'' at
% the top of the CV. For example:
%
%       \makeheading[\emph{Curriculum vitae}]{Your Name}
%
% will put an emphasized ``Curriculum vitae'' at the top of the document
% as a title. Likewise, a picture could be included:
%
%   \makeheading[{\includegraphics[height=1.5in]{my_picture}}]{Your Name}
%
% the picture will be flush right across from the name. For this example
% to work, make sure the extra set of curly braces is included. Also
% makes ure that \usepackage{graphicx} is somewhere in the preamble.
\newcommand{\makeheading}[2][]%
        {\hspace*{-\marginparsep minus \marginparwidth}%
         \begin{minipage}[t]{\textwidth+\marginparwidth+\marginparsep}%
             {\large \bfseries #2 \hfill #1}\\[-0.15\baselineskip]%
                 \rule{\columnwidth}{1pt}%
         \end{minipage}}

%%% SECTION HEADINGS

% The section headings. Flush left in small caps down pseudo-margin.
%
% Usage: \section{section name}
\renewcommand{\section}[1]{\pagebreak[3]%
    \vspace{1.3\baselineskip}%
    \phantomsection\addcontentsline{toc}{section}{#1}%
    \noindent\llap{\scshape\smash{\parbox[t]{\marginparwidth}{\hyphenpenalty=10000\raggedright #1}}}%
    \vspace{-\baselineskip}\par}

%%% LISTS

% This macro alters a list by removing some of the space that follows the list
% (is used by lists below)
\newcommand*\fixendlist[1]{%
    \expandafter\let\csname preFixEndListend#1\expandafter\endcsname\csname end#1\endcsname
    \expandafter\def\csname end#1\endcsname{\csname preFixEndListend#1\endcsname\vspace{-0.6\baselineskip}}}

% These macros help ensure that items in outer-type lists do not get
% separated from the next line by a page break
% (they are used by lists below)
\let\originalItem\item
\newcommand*\fixouterlist[1]{%
    \expandafter\let\csname preFixOuterList#1\expandafter\endcsname\csname #1\endcsname
    \expandafter\def\csname #1\endcsname{\let\oldItem\item\def\item{\pagebreak[2]\oldItem}\csname preFixOuterList#1\endcsname}
    \expandafter\let\csname preFixOuterListend#1\expandafter\endcsname\csname end#1\endcsname
    \expandafter\def\csname end#1\endcsname{\let\item\oldItem\csname preFixOuterListend#1\endcsname}}
\newcommand*\fixinnerlist[1]{%
    \expandafter\let\csname preFixInnerList#1\expandafter\endcsname\csname #1\endcsname
    \expandafter\def\csname #1\endcsname{\let\oldItem\item\let\item\originalItem\csname preFixInnerList#1\endcsname}
    \expandafter\let\csname preFixInnerListend#1\expandafter\endcsname\csname end#1\endcsname
    \expandafter\def\csname end#1\endcsname{\csname preFixInnerListend#1\endcsname\let\item\oldItem}}

% An itemize-style list with lots of space between items
%
% Usage:
%   \begin{outerlist}
%       \item ...    % (or \item[] for no bullet)
%   \end{outerlist}
\newlist{outerlist}{itemize}{3}
    \setlist[outerlist]{label=\enskip\textbullet,leftmargin=*}
    \fixendlist{outerlist}
    \fixouterlist{outerlist}

% An environment IDENTICAL to outerlist that has better pre-list spacing
% when used as the first thing in a \section
%
% Usage:
%   \begin{lonelist}
%       \item ...    % (or \item[] for no bullet)
%   \end{lonelist}
\newlist{lonelist}{itemize}{3}
    \setlist[lonelist]{label=\enskip\textbullet,leftmargin=*,partopsep=0pt,topsep=0pt}
    \fixendlist{lonelist}
    \fixouterlist{lonelist}

% An itemize-style list with little space between items
%
% Usage:
%   \begin{innerlist}
%       \item ...    % (or \item[] for no bullet)
%   \end{innerlist}
\newlist{innerlist}{itemize}{3}
    \setlist[innerlist]{label=\enskip\textbullet,leftmargin=*,parsep=0pt,itemsep=0pt,topsep=0pt,partopsep=0pt}
    \fixinnerlist{innerlist}

% An environment IDENTICAL to innerlist that has better pre-list spacing
% when used as the first thing in a \section
%
% Usage:
%   \begin{loneinnerlist}
%       \item ...    % (or \item[] for no bullet)
%   \end{loneinnerlist}
\newlist{loneinnerlist}{itemize}{3}
    \setlist[loneinnerlist]{label=\enskip\textbullet,leftmargin=*,parsep=0pt,itemsep=0pt,topsep=0pt,partopsep=0pt}
    \fixendlist{loneinnerlist}
    \fixinnerlist{loneinnerlist}

%%% EXTRA SPACE

% To add some paragraph space between lines.
% This also tells LaTeX to preferably break a page on one of these gaps
% if there is a needed pagebreak nearby.
\newcommand{\blankline}{\quad\pagebreak[3]}
\newcommand{\halfblankline}{\quad\vspace{-0.5\baselineskip}\pagebreak[3]}

%%% FORMATTING MACROS

% Provides a linked \doi{#1} that links doi:#1 to http://dx.doi.org/#1
\usepackage{doi}
% To change the text before the DOI, adjust this command
%\renewcommand\doitext{doi:}

% Provides a linked \url{#1} that doesn't require escape characters
\usepackage{url}

% You can adjust the style \url{} uses here:
% (options are: same, rm, sf, tt; defaults to tt)
\urlstyle{same}

% For \email{ADDRESS}, links ADDRESS to the url mailto:ADDRESS
% (uncomment to typeset the e\-/mail address in typewriter font;
%  otherwise, will be typeset in the \urlstyle above)
%\DeclareUrlCommand\emaillink{\urlstyle{tt}}
\providecommand*\emaillink[1]{\nolinkurl{#1}}
\providecommand*\email[1]{\href{mailto:#1}{\emaillink{#1}}}

\providecommand\BibTeX{{B\kern-.05em{\sc i\kern-.025em b}\kern-.08em \TeX}}
\providecommand\Matlab{\textsc{Matlab}}

% Custom hyphenation rules for words that LaTeX has trouble with
\hyphenation{bio-mim-ic-ry bio-in-spi-ra-tion re-us-a-ble pro-vid-er Media-Wiki}

% Provides the \nth{3} command so we can write numbers with superscripts
\usepackage[super]{nth}

\usepackage{relsize}
\newcommand\CC{C\nolinebreak[4]\hspace{-.05em}\raisebox{.4ex}{\relsize{-3}{\textbf{++}}}}
\newcommand\SVN{Subversion}
\newcommand\INCRTCLTK{[incr\space\nolinebreak Tcl/Tk]}
\newcommand\HUBZERO{\href{https://hubzero.org}{HUBzero}}
\newcommand\HUBZEROPLATFORM{\href{https://hubzero.org}{HUBzero Platform for Scientific Collaboration}}
\newcommand\RAPPTURE{\href{http://rappture.org}{Rappture Toolkit}}
\newcommand\NCN{\href{https://nanohub.org/groups/ncn}{Network for Computational Nanotechnology}}


%%%%%%%%%%%%%%%%%%%%%%%% End Helper Commands %%%%%%%%%%%%%%%%%%%%%%%%%%%

%%%%%%%%%%%%%%%%%%%%%%%%% Begin CV Document %%%%%%%%%%%%%%%%%%%%%%%%%%%%

\begin{document}
\thispagestyle{plain}
\makeheading[\emph{Curriculum vitae}]{Derrick Kearney}

\section{Contact Information}

% NOTE: Mind where the & separators and \\ breaks are in the following
%       table. Table is one row made up of three parboxes. The left
%       parbox has address info, the middle parbox has a vertical bar,
%       and the right parbox has phone and electronic contact
%       information.
%
% MACROS: \rcollength is the width of the right column of the table
%             (adjust it to your liking; default is 1.85in).
%         \spacewidth is width of area between left and right boxes.
%
\newlength{\rcollength}\setlength{\rcollength}{1.85in}%
\newlength{\spacewidth}\setlength{\spacewidth}{20pt}
%
\begin{tabular}[t]{@{}p{\textwidth-\rcollength-\spacewidth}@{}p{\spacewidth}@{}p{\rcollength}}%

% Address box
\parbox{\textwidth-\rcollength-\spacewidth}{%
%\href{http://www.hubzero.org/}{HUBzero}\\
%\href{http://www.rcac.purdue.edu/}{Research Computing, Information Technology at Purdue (IT@P), Purdue University}\\
IT@P Research Computing\\
Purdue University\\
155 South Grant St\\
West Lafayette, Indiana 47907}

&
% Uncomment to add a vertical bar in middle of contact information
%{\vrule width 0.5pt}
\parbox[m][5\baselineskip]{\spacewidth}{} &

% Non-snail-mail contact information
\parbox{\rcollength}{%
%\emph{Work:} +1-765-494-3659 \\
\emph{E-mail:} \email{dsk@purdue.edu}\\
\emph{WWW:} \href{https://github.com/codedsk}{github.com/codedsk}}

\end{tabular}


\section{Interests}
    Experienced in Linux based scientific software development.
    Interested in working with mobile operating systems and developing
    C/\CC, Python, and Rust libraries, while evangelizing
    about the benefits of software automation and open source technologies.

\section{Availability}

\begin{loneinnerlist}
    \item Start time is negotiable; may be possible to start immediately
    \item Geographic location is flexible, but there is preference
        for Lafayette, Indiana or Chicago, Illinois
\end{loneinnerlist}


\section{Education}

\href{http://www.purdue.edu/}{\textbf{Purdue University}},
West Lafayette, IN
\begin{outerlist}

\item[] M.S.,
        \href{https://engineering.purdue.edu/ECE}
             {Electrical and Computer Engineering}, May 2015
        \begin{innerlist}
            \item Thesis Topic:
                  \href{http://docs.lib.purdue.edu/dissertations/AAI1597870/}
                       {\emph{Automated testing in multimodal systems}}
            \item Adviser:
                  \href{https://engineering.purdue.edu/~smidkiff/}
                       {Professor Samuel P. Midkiff}
            \item Area of Study: Computer Engineering / Software Systems
        \end{innerlist}

\item[] B.S.,
        \href{https://engineering.purdue.edu/ECE}
             {Electrical and Computer Engineering}, Dec 2003
        \begin{innerlist}
            \item Computer Engineering specialization (emphasis on software systems)
            \item Minor in \href{http://www.purdue.edu/hhs/psy/} {Psychology}
        \end{innerlist}

\end{outerlist}

\section{Professional Experience}

\href{http://www.purdue.edu/}{\textbf{Purdue University}},
West Lafayette, IN
\begin{outerlist}

    \item[] \textit{Software Engineer}%
            \hfill \textbf{February 2005 to present}
    \item[] HUBzero\textsuperscript{\textregistered} Platform for Scientific Collaboration \\
        \url{https://hubzero.org}

        \begin{innerlist}

            \item Developer for the
                \href{http://rappture.org}{Rappture Toolkit}, a software
                library designed to help scientists rapidly assemble and
                deploy graphical user interfaces for their simulation codes.
                Contributed to several of the (incr)Tcl/Tk widgets and built
                languange bindings for C, \CC, FORTRAN, \Matlab, Octave,
                Perl, Python, and R. \\
                \url{http://rappture.org}

            \item Creator and maintainer of HUBcheck,
                a Python library used to build automation scripts
                and user level tests for HUBzero based websites and
                workspace environments. Currently being used by the HUBzero
                team to help perform automated nightly testing and
                software builds. \\
                \url{https://github.com/codedsk/hubcheck}

            \item Integrating Geospacial Analysis Building Blocks, GABBS,
                into the Rappture Toolkit and HUBzero Platform. Collaborating
                with a team of visualization and mapping experts
                to build Rappture Toolkit widgets that allow scientists to
                display, and end users to manipulate, large data sets through
                interactive maps. \\
                \url{https://mygeohub.org/groups/gabbs}

            \item Scientific application development for various
                research groups affiliated with Purdue University
                and the Network for Computational Nanotechnology.
                \hfill Recent applications include:
                \begin{innerlist}
                    \item[] SolarPV
                        \begin{innerlist}
                            \item[] Simulates electricity demand in residential
                              communities with solar photovoltaic (PV) systems.
                              The model takes into account the seasonal data for
                              different regions of the United States, appliance
                              profiles, and photovoltaic system profiles. It
                              returns an hourly and daily analysis of the
                              electric demand. Worked with a Chemical Engineering
                              based team to adapt the original proprietary model
                              to run on the HUBzero Platform, in an environment
                              where students could easily create and upload their own
                              data profiles for use by the model.
                            \item[] Coded in Java, MySQL, Python, Bash
                            \item[] \doi{10.4231/D3BV79W4T}
                        \end{innerlist}
                    \item[] ParticleVE
                        \begin{innerlist}
                            \item[] Track and estimate particle velocities
                              using video from the 2010 Deepwater Horizon
                              oil spill in the Gulf of Mexico. Worked with
                              a microfluidics and particle image velocimetry
                              expert to build this application, which uses
                              video from the oil spill and algebra to
                              assists users in estimating the amount of oil
                              that was released into the Gulf after the
                              drilling rig explosion.
                            \item[] Coded in Tcl/Tk, C (libav/ffmpeg), HTML
                            \item[] \doi{10.4231/D35D8NF30}
                        \end{innerlist}
                    \item[] NanoFET
                        \begin{innerlist}
                            \item[] Simulate ballistic transport in 2D MOSFET
                              devices. This application simulates the effects
                              of downscaling conventional CMOS devices,
                              uncovering the challenges of working on the
                              evershrinking nanoelectronics that run our
                              devices these days. Worked with a small team of
                              Electrical Engineering postdocs to parallelize
                              the code using MPI, set it up to run on TeraGrid
                              based supercomputing platforms, and deploy it
                              on nanohub.org as a community software tool.
                            \item[] Coded in FORTRAN
                            \item[] \doi{10254/nanohub-r1090.5}
                        \end{innerlist}
                    %\item[] Quantum Point Contact
                    %    \begin{innerlist}
                    %        \item[] Simulates the conductance and associated wavefunctions of Quantum Point Contacts
                    %        \item[] \url{http://nanohub.org/resources/qpc}
                    %    \end{innerlist}
                \end{innerlist}

            \item Software development and maintenance of virtual
                linux containers on HUBs managed by the HUBzero team,
                supporting over 400,000 registered users and over a
                million visitors per year, on sites like:
                \begin{innerlist}
                    \item[] \href{https://nanohub.org}{nanohub.org}
                        - nanotechnology research
                    \item[] \href{https://nees.org}{nees.org}
                        - earthquake, geophysics research
                    \item[] \href{https://pharmahub.org}{pharmahub.org}
                        - pharmaceutical production
                    \item[] \href{https://vhub.org}{vhub.org}
                        - volcano research and risk mitigatioon
                    \item[] \href{https://ccehub.org}{ccehub.org}
                        - cancer care engineering
                    \item[] \href{https://mygeohub.org}{mygeohub.org}
                        - geospacial modeling and data analysis
                \end{innerlist}

            \item Software development adviser for students
                participating in the Summer Undergraduate Research Fellowship
                (SURF) program. Work with other faculty, staff and graduate
                mentors to introduce students to nanotechnology research and
                software development best practices. \\
                \url{http://www.purdue.edu/surf}

        \end{innerlist}
\end{outerlist}

\halfblankline

\href{http://www.convergys.com/}{\textbf{Convergys Corporation}},
Cincinnati, OH
\begin{outerlist}

    \item[] \textit{Associate Programmer / Analyst}%
            \hfill \textbf{January 2004 to January 2005}\\
            \textit{Intern}%
            \hfill \textbf{May 2003 to August 2003}\\
            \textit{Intern}%
            \hfill \textbf{December 2002 to January 2003}\\
            \textit{Intern}%
            \hfill \textbf{May 2002 to August 2002}\\
            \begin{innerlist}
                \item Provided application support for Mediation Manager,
                    the company's flagship mobile phone billing and rating
                    software.

                \item Custom software design and development for Verizon
                    Wireless including onsite consultation regarding
                    system setup, testing, and operation.

                \item Provided production support for Cincinnati Bell
                    Wireless.
            \end{innerlist}

\end{outerlist}


\section{Refereed Journal Publications}

\begin{bibenum}

    \item McLennan, M., Clark, S., Deelman, E., Rynge, M., Vahi, K.,
          McKenna, F., Kearney, D., Song, C. (2015). HUBzero and Pegasus:
          integrating scientific workflows into science gateways.
          \emph{Concurrency and Computation: Practice and Experience},
          27(2), 328-343.\\
          \doi{10.1002/cpe.3257}

    \item McLennan, M., Clark, S., Deelman, E., Rynge, M., Vahi, K.,
          McKenna, F., Kearney, D., Song, C. (2013). Bringing scientific
          workflow to the masses via pegasus and hubzero.
          \emph{parameters}, 13, 14.

    \item Ahmed, S., Klimeck, G., Kearney, D., McLennan, M., \& Anantram,
          M. P. (2007). Quantum Simulations of Dual Gate MOSFET Devices:
          Building and Deploying Community Nanotechnology Software Tools
          on nanoHUB.org. \emph{International Journal of High Speed
          Electronics and Systems}, 17(03), 485-494.\\
          \doi{10.1142/S0129156407004679}

\end{bibenum}

\vspace{0.1in}

\section{Conference Proceedings}

\begin{bibenum}

    \item GL Rochon, Carol Xiaohui Song, Lan Zhao, Dev Niyogi, Derrick S. Kearney and
          Jie Shan. "Supporting Interdisciplinary e-Science through Real-Time Remote
          Sensing \& High Performance Computing in a High Bandwidth Environment," 20th
          International CODATA Conference. Scientific Data and Knowledge within the
          Information Society. Oct. 23-25, 2006, Beijing, China.

    \item GL Rochon, J. Paul Robinson, Carol Xiaohui Song, Lan Zhao, Dev Niyogi and
          Derrick S. Kearney. "Remote and Proximal Sensing in Support of Disaster
          Mitigation and Sustainable Development," Chinese Academy of Sciences, China
          Remote Sensing Ground Station (RSGS), Beijing, China, Oct. 24, 2006.

    \item GL Rochon, Mohamed A. Mohamed, Dev Niyogi, Melba M. Crawford, J. Paul
          Robinson, Souleymane Fall, Joseph E. Quansah, Larry Biehl, Jie Shan, Carol
          Xiaohui Song, Derrick S. Kearney, Lan Zhao and Amy Neuenschwander.
          "Integration of Real-Time and Archival Remote Sensing with High Performance
          Computing \& Dynamic Modeling to Support Disaster Mitigation." African
          Association for Remote Sensing of Environment (AARSE), Cairo, Egypt, Oct. 30
          - Nov. 2, 2006.

    \item GL Rochon, Mohamed A. Mohamed, Dev Niyogi, Melba M. Crawford, J. Paul
          Robinson, Souleymane Fall, Joseph E. Quansah, Larry Biehl, Jie Shan, Carol
          Xiaohui Song, Derrick S. Kearney, Lan Zhao and Amy Neuenschwander. 'In Situ
          Monitoring, Satellite Remote Sensing, Proximal Sensing \& High Performance
          Computing to Facilitate Sustainability and Disaster Preparedness." Cairo
          University, Aerospace Studies Department, Cairo, Egypt, Nov. 2, 2006.

    \item GL Rochon, Mohamed A. Mohamed, Dev Niyogi, Melba M. Crawford, J. Paul
          Robinson, Souleymane Fall, Joseph E. Quansah, Larry Biehl, Jie Shan, Carol
          Xiaohui Song, Derrick S. Kearney, Lan Zhao and Amy Neuenschwander.
          "Multidisciplinary Research Enabled by Real-time and Archival Remote Sensing
          and High Performance Computing to Support Environmental Sustainability and
          Disaster Mitigation." Al Ahzar University, Departments of Astronomy and
          Meteorology, Cairo, Egypt, Nov. 4, 2006.

\end{bibenum}

\vspace{0.1in}

\section{Conference Talks}

\begin{bibenum}

    \item Kearney, D. Supercharge your simulation tools with nanoHUB.org.
        \emph{Nanoindentation 2015}, Urbana, Illinois, April 1--2, 2015. \\
        \url{https://nanohub.org/groups/nanobio/nano_agenda_april_2015}
%        \url{https://nanohub.org/groups/nanobio/nanoindentationworkshop}
%        \url{https://nanohub.org/resources/22809}
%       introduction to nanohub talk
%       20 people, 30 minutes

    \item Kearney, D. Design Patterns beyond the Page Object: An
        investigation into the design patterns used while building
        page objects. \emph{Selenium Conf '14}, Bangalore, India,
        September 4--6, 2014. \\
        \url{https://www.youtube.com/watch?v=AVrnBJDQeaI}
%       describe 3 newly formalized design patterns, used within HUBcheck,
%       that make Selenium based web page automation easier.
%       250 people, 1 hour talk

    \item Kearney, D. HUBcheck - Check the Hub. \emph{HUBbub 2011},
        Indianapolis, Indiana, April 07--08, 2011. \\
        \url{https://hubzero.org/resources/422}

\end{bibenum}

\halfblankline

\section{Workshops and Teaching Experience}

\begin{bibenum}

    \item Kearney, D. Rappture Toolkit, Submit for Job Submission, and
        PUQ for Uncertainty Quantification. \emph{HUBbub 2015},
        Indianapolis, Indiana, September 13--16, 2015. \\
        \url{https://hubzero.org/hubbub/2015/}
%       1/2 day workshop showing off the Rappture Toolkit for building
%       simulation tools, USC's Pegasus for job management on clusters,
%       Submit for job submission and performing sweeps, and PUQ for
%       Uncertainty Quantification.
%       4 hours, 10 people

    \item Kearney, D. Introduction to Workspaces and Simulation Tool
        Development on nanoHUB.org. \emph{nanoHUB User Conference 2015},
        West Lafayette, Indiana, August 31--September 1, 2015. \\
        \url{https://nanohub.org/groups/conference}
%        \url{https://nanohub.org/groups/conference/2015conference}
%       3 hours, 3 presentations, 3 exercises
%       what is a workspace
%       intro to rappture
%       uploading and publishing tools

    \item Kearney, D. Software Bootcamp. \emph{NCN Software Bootcamp 2015},
        West Lafayette, Indiana, June 1--3, 2015.
%        \url{https://nanohub.org/groups/conference/2015conference}
%       3 day workshop introducing surf students to nanoHUB, Rappture Toolkit
%       and good software practices, 30 people

    \item Kearney, D. Introduction to the Rappture Toolkit.
        \emph{HUBbub 2014}, Indianapolis, Indiana, September 28--October 1, 2014. \\
        \url{https://hubzero.org/resources/1216}
%        \url{https://hubzero.org/hubbub/2014/}
%       1 day (2 half day) workshop introducing HUBbub participants to Online
%       simulation, HUBzero Workspaces, and Rappture Toolkit.
%       10 people

    \item Kearney, D. Software Bootcamp.
        \emph{NCN Software Bootcamp (UC Merced Edition)}, Merced, California,
        July 14--18, 2014.
%        \url{https://hubzero.org/hubbub/2014/}
%       5 half day workshop introducing UC Merced students to building and
%       using online simulation tools on nanoHUB.org, working with them to
%       learn programming and how it can be applied to their research.
%       10 people

\end{bibenum}

\halfblankline

\section{Service}

    \href{http://bit.ly/caribsa/}{\textbf{Caribbean Student Association (CARIBSA)}}, Purdue University
    \begin{outerlist}

        \item[] Adviser %
                \hfill \textbf{August 2005 to Present}


                Guide students through the acquisition of grants and
                resources to further the club's mission of educating
                the community about the people, culture, and
                contributions of islands and nations in the Caribbean
                region. \\

                Recent club activities include:
                \begin{innerlist}
                    \item Bob Marley Birthday Bash % \hfill Feb 2016
                    \item Homecoming Parade % \hfill Nov 2015
                    \item Lafayette Transitional Housing Center Dinner % \hfill Fall 2010
                    \item Help For Haiti Earthquake Relief Fundraisers  % \hfill Spring 2010
                    % \item Lyn Treece Boys and Girls Club Carnival % \hfill Spring 2007
                    % \item Hanna Community Center Homework Assistance Program % \hfill Fall 2007
                    % \item Hanna Community Center Carnival % \hfill Fall 2006
                \end{innerlist}

                \url{bit.ly/caribsa} \\
                \url{bit.ly/caribsa-facebook}

    \end{outerlist}

    \halfblankline

    \href{http://www.lthc.net/}{\textbf{Lafayette Transitional Housing Center}}
    \begin{outerlist}

        \item[] Volunteer Chef %
                \hfill \textbf{November 2012 to April 2015}

                \begin{innerlist}
                    \item Developed menus, sourced food, and cooked monthly dinners for 15 to 60 clients
                \end{innerlist}

    \end{outerlist}

    \halfblankline

    \textbf{Boy Scout Troop 336}, West Lafayette, Indiana
    \begin{outerlist}

        \item[] Committee Chair %
                \hfill \textbf{February 2012 to December 2013}

                \begin{innerlist}
                    \item Helped launch and charter a new troop, assisting one scout to
                          reach the rank of Eagle Scout and nearly all other scouts to reach
                          the rank of First Class.
                    \item Built a team, from parents and members of the Purdue University
                          College of Technology, to provide enrichment for the troop through
                          grant funded activities that supported scout rank advancement.
                    \item Conducted monthly information sessions and
                          collaborated with parents to support the external
                          needs of the troop like transportation to out of
                          town destinations, camping permits, and equipment
                          acquisition.
                    \item Worked with scouts to implement a leadership team
                          responsible for organizing trip meals, managing troop equipment
                          after trips, teaching incomming scouts new skills, and upholding
                          Scouting virtues.
                \end{innerlist}

    \end{outerlist}

    \halfblankline

    \href{http://www.purdue.edu/mep/}{\textbf{Minority Engineering Program (MEP)}}, Purdue University
    \begin{outerlist}

        \item[] Summer Programs Volunteer %
                \hfill \textbf{February 2005 to August 2008}

                \begin{innerlist}
                    \item Organized lab tours of Birck Nanotechnology
                          and Bindley Bioscience Centers.
                    \item Coordinated with the Network for Computational
                          Nanotechnology and Discovery Park to produce
                          the initial Birck Nanotechnology Center atrium
                          exhibit.
                    \item Developed and taught
                          \textit{Top Down vs. Bottom Up: Intro to Photolithography},
                          a nanotechnology based shortcourse to suppliment
                          the experience of \nth{6}--\nth{10} grade MEP
                          summer program students.
                \end{innerlist}

    \end{outerlist}


\halfblankline


\section{Awards}
%
    \begin{innerlist}
        \item Motorola/Purdue Electrical and Computer Engineering Scholarship, 2003
        \item Minority Engineering Excellence Scholarship, 2002
        \item General Motors Academic Excellence Scholarship, 2001
        \item Lockheed-Martin Academic Excellence Scholarship, 2000
        \item Purdue University Black Caucus Award, 2000
        \item Eagle Scout, 1998
    \end{innerlist}

\halfblankline

\section{Skills}
%
    Computer Programming:
    %
    \begin{innerlist}
        \item C, \CC, FORTRAN, Perl, Python, Tcl/Tk,
            UNIX shell, GNU Tools (Autotools, configure,
            Make, gdb, gnu compilers and friends),
            SQL, HTML
    \end{innerlist}

    \halfblankline

    Numerical Analysis:
    %
    \begin{innerlist}
        \item \Matlab, Octave, R
    \end{innerlist}

    \halfblankline

    Version Control and Software Configuration Management:
    %
    \begin{innerlist}
        \item Git, Subversion
    \end{innerlist}

    \halfblankline

    Operating Systems:
    %
    \begin{innerlist}
        \item Debian Linux, Ubuntu
    \end{innerlist}

\halfblankline

\section{More Information}

    Auxiliary documents and references available upon request.

\end{document}

%%%%%%%%%%%%%%%%%%%%%%%%%% End CV Document %%%%%%%%%%%%%%%%%%%%%%%%%%%%%

%----------------------------------------------------------------------%
% The following is copyright and licensing information for
% redistribution of this LaTeX source code; it also includes a liability
% statement. If this source code is not being redistributed to others,
% it may be omitted. It has no effect on the function of the above code.
%----------------------------------------------------------------------%
% Copyright (c) 2007, 2008, 2009, 2010, 2011 by Theodore P. Pavlic
%
% Unless otherwise expressly stated, this work is licensed under the
% Creative Commons Attribution-Noncommercial 3.0 United States License. To
% view a copy of this license, visit
% http://creativecommons.org/licenses/by-nc/3.0/us/ or send a letter to
% Creative Commons, 171 Second Street, Suite 300, San Francisco,
% California, 94105, USA.
%
% THE SOFTWARE IS PROVIDED "AS IS", WITHOUT WARRANTY OF ANY KIND, EXPRESS
% OR IMPLIED, INCLUDING BUT NOT LIMITED TO THE WARRANTIES OF
% MERCHANTABILITY, FITNESS FOR A PARTICULAR PURPOSE AND NONINFRINGEMENT.
% IN NO EVENT SHALL THE AUTHORS OR COPYRIGHT HOLDERS BE LIABLE FOR ANY
% CLAIM, DAMAGES OR OTHER LIABILITY, WHETHER IN AN ACTION OF CONTRACT,
% TORT OR OTHERWISE, ARISING FROM, OUT OF OR IN CONNECTION WITH THE
% SOFTWARE OR THE USE OR OTHER DEALINGS IN THE SOFTWARE.
%----------------------------------------------------------------------%
