%%%%%%%%%%%%%%%%%%%%%%%%%%%%%%%%%%%%%%%%%%%%%%%%%%%%%%%%%%%%%%%%%%%%%%%%
%%%%%%%%%%%%%%%%%%%%%% Simple LaTeX CV Template %%%%%%%%%%%%%%%%%%%%%%%%
%%%%%%%%%%%%%%%%%%%%%%%%%%%%%%%%%%%%%%%%%%%%%%%%%%%%%%%%%%%%%%%%%%%%%%%%

%%%%%%%%%%%%%%%%%%%%%%%%%%%%%%%%%%%%%%%%%%%%%%%%%%%%%%%%%%%%%%%%%%%%%%%%
%% NOTE: If you find that it says                                     %%
%%                                                                    %%
%%                           1 of ??                                  %%
%%                                                                    %%
%% at the bottom of your first page, this means that the AUX file     %%
%% was not available when you ran LaTeX on this source. Simply RERUN  %%
%% LaTeX to get the ``??'' replaced with the number of the last page  %%
%% of the document. The AUX file will be generated on the first run   %%
%% of LaTeX and used on the second run to fill in all of the          %%
%% references.                                                        %%
%%%%%%%%%%%%%%%%%%%%%%%%%%%%%%%%%%%%%%%%%%%%%%%%%%%%%%%%%%%%%%%%%%%%%%%%

%%%%%%%%%%%%%%%%%%%%%%%%%%%% Document Setup %%%%%%%%%%%%%%%%%%%%%%%%%%%%

% Don't like 10pt? Try 11pt or 12pt
\documentclass[10pt]{article}
\RequirePackage[T1]{fontenc}

% The automated optical recognition software used to digitize resume
% information works best with fonts that do not have serifs. This
% command uses a sans serif font throughout. Uncomment both lines (or at
% least the second) to restore a Roman font (i.e., a font with serifs).
\usepackage{times}
\renewcommand{\familydefault}{\sfdefault}

% The OCR software also has a hard time with italics. These commands get
% rid of the two common ways to italicize text in LaTeX. Get rid of them
% to turn italics back on.
\renewcommand\emph[1]{#1}
\renewcommand\textit[1]{\underline{\smash{#1}}}

% This is a helpful package that puts math inside length specifications
\usepackage{calc}

% This package helps LaTeX auto-hyphenate hyphenated words if you use
% special hyphens. For example, bio\-/mimicry will properly hyphenate
% ``mimicry'' if necessary.
\usepackage[shortcuts]{extdash}

% Layout: Puts the section titles on left side of page
\reversemarginpar

%
%         PAPER SIZE, PAGE NUMBER, AND DOCUMENT LAYOUT NOTES:
%
% The next \usepackage line changes the layout for CV style section
% headings as marginal notes. It also sets up the paper size as either
% letter or A4. By default, letter was used. If A4 paper is desired,
% comment out the letterpaper lines and uncomment the a4paper lines.
%
% As you can see, the margin widths and section title widths can be
% easily adjusted.
%
% ALSO: Notice that the includefoot option can be commented OUT in order
% to put the PAGE NUMBER *IN* the bottom margin. This will make the
% effective text area larger.
%
% IF YOU WISH TO REMOVE THE ``of LASTPAGE'' next to each page number,
% see the note about the +LP and -LP lines below. Comment out the +LP
% and uncomment the -LP.
%
% IF YOU WISH TO REMOVE PAGE NUMBERS, be sure that the includefoot line
% is uncommented and ALSO uncomment the \pagestyle{empty} a few lines
% below.
%

%% Use these lines for letter-sized paper
\usepackage[paper=letterpaper,
            %includefoot, % Uncomment to put page number above margin
            marginparwidth=1.2in,     % Length of section titles
            marginparsep=.05in,       % Space between titles and text
            %margin=1in,               % 1 inch margins
            margin=0.73in,            % page margins in inches
            includemp]{geometry}

%% Use these lines for A4-sized paper
%\usepackage[paper=a4paper,
%            %includefoot, % Uncomment to put page number above margin
%            marginparwidth=30.5mm,    % Length of section titles
%            marginparsep=1.5mm,       % Space between titles and text
%            margin=25mm,              % 25mm margins
%            includemp]{geometry}

%% More layout: Get rid of indenting throughout entire document
\setlength{\parindent}{0in}

% Provides special list environments and macros to create new ones
\usepackage[shortlabels]{enumitem}

% Simpler bibsections for CV sections
% (thanks to natbib for inspiration)
%
% * For lists of references with hanging indents and no numbers:
%
%   \begin{bibsection}
%       \item ...
%   \end{bibsection}
%
% * For numbered lists of references (with hanging indents):
%
%   \begin{bibenum}
%       \item ...
%   \end{bibenum}
%
%   Note that bibenum numbers continuously throughout. To reset the
%   counter, use
%
%   \restartlist{bibenum}
%
%   at the place where you want the numbering to reset.

\makeatletter
\newlength{\bibhang}
\setlength{\bibhang}{1em}
\newlength{\bibsep}
 {\@listi \global\bibsep\itemsep \global\advance\bibsep by\parsep}
\newlist{bibsection}{itemize}{3}
\setlist[bibsection]{label=,leftmargin=\bibhang,%
        itemindent=-\bibhang,
        itemsep=\bibsep,parsep=\z@,partopsep=0pt,
        topsep=0pt}
\newlist{bibenum}{enumerate}{3}
\setlist[bibenum]{label=[\arabic*],resume,leftmargin={\bibhang+\widthof{[999]}},%
        itemindent=-\bibhang,
        itemsep=\bibsep,parsep=\z@,partopsep=0pt,
        topsep=0pt}
\let\oldendbibenum\endbibenum
\def\endbibenum{\oldendbibenum\vspace{-.6\baselineskip}}
\let\oldendbibsection\endbibsection
\def\endbibsection{\oldendbibsection\vspace{-.6\baselineskip}}
\makeatother

%%% Setup header and footer (with page number and possible last page)
%
% The first block sets up pages 2--end
% The second block sets up page 1 formatting
%
%%%
%
% NOTE: comment the +LP lines and uncomment the -LP lines to have page
%       numbers without the ``of ##'' last page reference)
%
% NOTE: uncomment the \pagestyle{empty} line to get rid of all page
%       numbers on pages 2--end. To get rid of page numbers on page 1,
%       comment out the \thispagestyle{plain} line on the first page
%       below.
%       (also make sure includefoot is commented out above)
%
\usepackage{fancyhdr,lastpage}
\pagestyle{fancy}
%\pagestyle{empty}      % Uncomment this to get rid of page numbers
\fancyhf{}\renewcommand{\headrulewidth}{0pt}
\fancyfootoffset{\marginparsep+\marginparwidth}
\newlength{\footpageshift}
\setlength{\footpageshift}
          {0.5\textwidth+0.5\marginparsep+0.5\marginparwidth-2in}

%%%% PAGES 2--9 NUMBERING:
%% These two lines put page number in upper-right corner of pages 2--end
\rhead{Kearney, p.~\arabic{page} of \protect\pageref*{LastPage}}   % +LP
%\rhead{Kearney, p.~\arabic{page}}                                 % -LP

%% These lines put page number in bottom (center) of pages 2--end
%\lfoot{\hspace{\footpageshift}%
%       \parbox{4in}{\, \hfill %
%                    \arabic{page} of \protect\pageref*{LastPage} % +LP
%%                    \arabic{page}                               % -LP
%                    \hfill \,}}
%%%% END PAGE 2--9 NUMBERING

%%%% PAGE 1 NUMBERING:
\makeatletter
\let\oldps@plain\ps@plain
\renewcommand{\ps@plain}{\oldps@plain%
\renewcommand{\@evenfoot}{\hspace*{-\footpageshift}\hfil %
    p.~\arabic{page} of \protect\pageref*{LastPage} % +LP
%    p.~\arabic{page}                               % -LP
    \hfil}%
\renewcommand{\@oddfoot}{\@evenfoot}}
\makeatother
%%%% END PAGE 1 NUMBERING

% Finally, give us PDF bookmarks and colored links
%
% NOTE: Some OCR software might be negatively affected by hyperlinks. So
%       most employers recommend the draft option here. Alternatively,
%       making all links black (as opposed to darkblue) should hopefully
%       prevent problems with most OCR.
%
% (to enable hyperlinks and bookmarks, comment out ``draft'' line;
%  to disable hyperlinks and bookmarks, uncomment ``draft'' line)
\usepackage{color,hyperref}
\definecolor{darkblue}{rgb}{0.0,0.0,0.3}
\hypersetup{breaklinks,colorlinks,
            linkcolor=black,urlcolor=black,
            anchorcolor=black,citecolor=black,
            %linkcolor=darkblue,urlcolor=darkblue,
            %anchorcolor=darkblue,citecolor=darkblue,
            %draft
            }

% allow user to include graphics like qrcodes
\usepackage{graphicx}


%%%%%%%%%%%%%%%%%%%%%%%% End Document Setup %%%%%%%%%%%%%%%%%%%%%%%%%%%%


%%%%%%%%%%%%%%%%%%%%%%%%%%% Helper Commands %%%%%%%%%%%%%%%%%%%%%%%%%%%%

%%% HEADING AT TOP OF CURRICULUM VITAE

% The title (name) with a horizontal rule under it
% (optional argument typesets an object right-justified across from name
%  as well)
%
% Usage: \makeheading{name}
%        OR
%        \makeheading[right_object]{name}
%
% Place at top of document. It should be the first thing.
% If ``right_object'' is provided in the square-braced optional
% argument, it will be right justified on the same line as ``name'' at
% the top of the CV. For example:
%
%       \makeheading[\emph{Curriculum vitae}]{Your Name}
%
% will put an emphasized ``Curriculum vitae'' at the top of the document
% as a title. Likewise, a picture could be included:
%
%   \makeheading[{\includegraphics[height=1.5in]{my_picture}}]{Your Name}
%
% the picture will be flush right across from the name. For this example
% to work, make sure the extra set of curly braces is included. Also
% makes sure that \usepackage{graphicx} is somewhere in the preamble.
\newcommand{\makeheading}[2][]%
        {\hspace*{-\marginparsep minus \marginparwidth}%
         \begin{minipage}[t]{\textwidth+\marginparwidth+\marginparsep}%
             {\large \bfseries #2 \hfill #1}\\[-0.15\baselineskip]%
                 \rule{\columnwidth}{1pt}%
         \end{minipage}}

\newcommand{\makeheadingqr}[3][]%
        {\hspace*{-\marginparsep minus \marginparwidth}%
         \begin{minipage}[t]{\textwidth+\marginparwidth+\marginparsep}%
             {\large \bfseries #2 \hfill #1 \includegraphics[scale=1]{#3}}\\[-0.15\baselineskip]%
                 \rule{\columnwidth}{1pt}%
         \end{minipage}}

%%% SECTION HEADINGS

% The section headings. Flush left in small caps down pseudo-margin.
%
% Usage: \section{section name}
\renewcommand{\section}[1]{\pagebreak[3]%
    \vspace{1.0\baselineskip}%
    \phantomsection\addcontentsline{toc}{section}{#1}%
    \noindent\llap{\scshape\smash{\parbox[t]{\marginparwidth}{\hyphenpenalty=10000\raggedright #1}}}%
    \vspace{-\baselineskip}\par}

%%% LISTS

% This macro alters a list by removing some of the space that follows the list
% (is used by lists below)
\newcommand*\fixendlist[1]{%
    \expandafter\let\csname preFixEndListend#1\expandafter\endcsname\csname end#1\endcsname
    \expandafter\def\csname end#1\endcsname{\csname preFixEndListend#1\endcsname\vspace{-0.6\baselineskip}}}

% These macros help ensure that items in outer-type lists do not get
% separated from the next line by a page break
% (they are used by lists below)
\let\originalItem\item
\newcommand*\fixouterlist[1]{%
    \expandafter\let\csname preFixOuterList#1\expandafter\endcsname\csname #1\endcsname
    \expandafter\def\csname #1\endcsname{\let\oldItem\item\def\item{\pagebreak[2]\oldItem}\csname preFixOuterList#1\endcsname}
    \expandafter\let\csname preFixOuterListend#1\expandafter\endcsname\csname end#1\endcsname
    \expandafter\def\csname end#1\endcsname{\let\item\oldItem\csname preFixOuterListend#1\endcsname}}
\newcommand*\fixinnerlist[1]{%
    \expandafter\let\csname preFixInnerList#1\expandafter\endcsname\csname #1\endcsname
    \expandafter\def\csname #1\endcsname{\let\oldItem\item\let\item\originalItem\csname preFixInnerList#1\endcsname}
    \expandafter\let\csname preFixInnerListend#1\expandafter\endcsname\csname end#1\endcsname
    \expandafter\def\csname end#1\endcsname{\csname preFixInnerListend#1\endcsname\let\item\oldItem}}

% An itemize-style list with lots of space between items
%
% Usage:
%   \begin{outerlist}
%       \item ...    % (or \item[] for no bullet)
%   \end{outerlist}
\newlist{outerlist}{itemize}{3}
    \setlist[outerlist]{label=\enskip\textbullet,leftmargin=*}
    \fixendlist{outerlist}
    \fixouterlist{outerlist}

% An environment IDENTICAL to outerlist that has better pre-list spacing
% when used as the first thing in a \section
%
% Usage:
%   \begin{lonelist}
%       \item ...    % (or \item[] for no bullet)
%   \end{lonelist}
\newlist{lonelist}{itemize}{3}
    \setlist[lonelist]{label=\enskip\textbullet,leftmargin=*,partopsep=0pt,topsep=0pt}
    \fixendlist{lonelist}
    \fixouterlist{lonelist}

% An itemize-style list with little space between items
%
% Usage:
%   \begin{innerlist}
%       \item ...    % (or \item[] for no bullet)
%   \end{innerlist}
\newlist{innerlist}{itemize}{3}
    \setlist[innerlist]{label=\enskip\textbullet,leftmargin=*,parsep=0pt,itemsep=0pt,topsep=0pt,partopsep=0pt}
    \fixinnerlist{innerlist}

% An environment IDENTICAL to innerlist that has better pre-list spacing
% when used as the first thing in a \section
%
% Usage:
%   \begin{loneinnerlist}
%       \item ...    % (or \item[] for no bullet)
%   \end{loneinnerlist}
\newlist{loneinnerlist}{itemize}{3}
    \setlist[loneinnerlist]{label=\enskip\textbullet,leftmargin=*,parsep=0pt,itemsep=0pt,topsep=0pt,partopsep=0pt}
    \fixendlist{loneinnerlist}
    \fixinnerlist{loneinnerlist}

%%% EXTRA SPACE

% To add some paragraph space between lines.
% This also tells LaTeX to preferably break a page on one of these gaps
% if there is a needed pagebreak nearby.
\newcommand{\blankline}{\quad\pagebreak[3]}
\newcommand{\halfblankline}{\quad\vspace{-0.5\baselineskip}\pagebreak[3]}

%%% FORMATTING MACROS

% Provides a linked \doi{#1} that links doi:#1 to http://dx.doi.org/#1
\usepackage{doi}
% To change the text before the DOI, adjust this command
%\renewcommand\doitext{doi:}

% Provides a linked \url{#1} that doesn't require escape characters
\usepackage{url}

% You can adjust the style \url{} uses here:
% (options are: same, rm, sf, tt; defaults to tt)
\urlstyle{same}

% For \email{ADDRESS}, links ADDRESS to the url mailto:ADDRESS
% (uncomment to typeset the e\-/mail address in typewriter font;
%  otherwise, will be typeset in the \urlstyle above)
%\DeclareUrlCommand\emaillink{\urlstyle{tt}}
\providecommand*\emaillink[1]{\nolinkurl{#1}}
\providecommand*\email[1]{\href{mailto:#1}{\emaillink{#1}}}

\providecommand\BibTeX{{B\kern-.05em{\sc i\kern-.025em b}\kern-.08em \TeX}}
\providecommand\Matlab{\textsc{Matlab}}

% Provides the \nth{3} command so we can write numbers with superscripts
\usepackage[super]{nth}

\usepackage{relsize}
\newcommand\CC{C\nolinebreak[4]\hspace{-.05em}\raisebox{.4ex}{\relsize{-3}{\textbf{++}}}}
\newcommand\SVN{Subversion}
\newcommand\INCRTCL{[incr\space\nolinebreak Tcl]}
\newcommand\INCRTCLTK{[incr\space\nolinebreak Tcl/Tk]}
\newcommand\TCLTK{Tcl/Tk}
\newcommand\HUBZERO{\href{https://hubzero.org}{HUBzero}}
\newcommand\HUBZEROPLATFORM{\href{https://hubzero.org}{HUBzero Platform}}
\newcommand\HUBZEROREGISTERED{\href{https://hubzero.org}{HUBzero\textsuperscript{\textregistered}}}
\newcommand\HUBZEROPLATFORMSCICOL{\href{https://hubzero.org}{HUBzero Platform for Scientific Collaboration}}
\newcommand\RAPPTURE{\href{http://rappture.org}{Rappture Toolkit}}
\newcommand\NCN{\href{https://nanohub.org/groups/ncn}{Network for Computational Nanotechnology}}

%%%%%%%%%%%%%%%%%%%%%%%% End Helper Commands %%%%%%%%%%%%%%%%%%%%%%%%%%%

%%%%%%%%%%%%%%%%%%%%%%%%% Begin CV Document %%%%%%%%%%%%%%%%%%%%%%%%%%%%

\begin{document}
\thispagestyle{plain}
\makeheadingqr[]{Derrick Kearney}{qrcoderesume.pdf}

\section{Contact Information}

% NOTE: Mind where the & separators and \\ breaks are in the following
%       table. Table is one row made up of three parboxes. The left
%       parbox has address info, the middle parbox has a vertical bar,
%       and the right parbox has phone and electronic contact
%       information.
%
% MACROS: \rcollength is the width of the right column of the table
%             (adjust it to your liking; default is 1.85in).
%         \spacewidth is width of area between left and right boxes.
%
\newlength{\rcollength}\setlength{\rcollength}{2.2in}%
\newlength{\spacewidth}\setlength{\spacewidth}{20pt}
%
\begin{tabular}[t]{@{}p{\textwidth-\rcollength-\spacewidth}@{}p{\spacewidth}@{}p{\rcollength}}%
% Address box
\parbox{\textwidth-\rcollength-\spacewidth}{%

Lafayette, IN \\
United States of America}

&

% Uncomment to add a vertical bar in middle of contact information
%{\vrule width 0.5pt}
\parbox[m][2\baselineskip]{\spacewidth}{}

&

% Non-snail-mail contact information
\parbox{\rcollength}{%
\emph{E-mail:} \email{x646b21@gmail.com}\\
\emph{GitHub:} \url{https://github.com/codedsk}\\
\emph{Skype:} ij28ms}

\end{tabular}


\section{Interests}

    Experienced in community oriented, open source, scientific software library
    and toolkit development. A quick learner interested in leveraging new
    technologies to create software that impacts mobile and multicore platforms
    at the operating system and ecosystem levels, while evangelizing about the
    benefits of testing, automation, and open source technologies.


\section{Contributions}

% 1. help drive rust adoption
% 2. work write rust language, tools, code
% 3. work on distributed team from around the world
% 4. work on open source code
% 5. help others in the community.


        Developer and primary simulation tools contact for \HUBZEROREGISTERED,
        \url{https://hubzero.org}, an open source platform for creating
        dynamic, production quality websites that support scientific research
        and educational activities with over 400,000 registered users and over
        a million visitors per year.

        \halfblankline

        \begin{innerlist}

% 1. help drive adoption
% 4. open source
% 5. help community
            \item \textbf{Engaged users and open source community members} to help
                increase adoption of both the \HUBZEROPLATFORM \space and \RAPPTURE,
                implement new features, develop automated testing
                strategies, and fix bugs. Resolved over 400 community
                interactions in our internal support ticket system.

            \halfblankline

% 2. write language, tools, code
% 4. open source
             \item \textbf{Designed language bindings and several core Tcl/Tk widgets}
                for the \RAPPTURE, an open source software library designed to help
                scientists rapidly assemble and deploy graphical user interfaces for
                their simulation codes. Responsibilities included writing and maintaining
                language bindings for C, \CC, FORTRAN, \Matlab, Octave, Perl, Python,
                and R. \\
                \url{http://rappture.org}


            \halfblankline

% 3. distributed teams
% 5. help community
            \item \textbf{Collaborated with dozens of research groups} affiliated with the
                \href{https://nanohub.org/groups/ncn}{Network for Computational Nanotechnology},
                \href{https://engineering.purdue.edu/ChE/People/ptProfile?id=12436}{Purdue},
                \href{http://nanobionode.illinois.edu}{UIUC},
                \href{http://pegasus.isi.edu}{UCSD},
                \href{http://telab.vuse.vanderbilt.edu/greg.walker}{Vanderbilt},
                \href{http://www.engr.siu.edu/staff1/ahmed/mywebpage/ahmed.html}{Southern Illinois},
                \href{http://faculty1.ucmerced.edu/amartini}{UC Merced},
                \href{http://cbiit.nci.nih.gov/ncip}{NIH},
                and other institutions around the world to build freely available,
                scientific applications including:
                \begin{outerlist}
% 1. help drive adoption
                    \item[] SolarPV
                        \begin{innerlist}
                            \item[] Simulates electricity demand in residential
                              communities with solar photovoltaic (PV) systems.
                              Worked with a Chemical Engineering team to adapt
                              a proprietary model to run on the \HUBZEROPLATFORM,
                              where students could easily create,
                              upload, and simulate using their own data.
                            \item[] Coded in Java, MySQL, Python, Bash
                            \item[] \url{https://nanohub.org/resources/solarpv} \hfill\doi{10.4231/D3BV79W4T}
                        \end{innerlist}
                    \item[] ParticleVE
                        \begin{innerlist}
                            \item[] Track and estimate particle velocities
                              using video from the 2010 Deepwater Horizon
                              oil spill in the Gulf of Mexico. Worked with
                              a microfluidics and particle image velocimetry
                              expert to build an open source application which
                              uses video from the oil spill and algebra to
                              assist users in estimating the amount of oil
                              released into the Gulf after the drilling rig explosion.
                            \item[] Coded in Tcl/Tk, C (libav/ffmpeg), HTML
                            \item[] \url{https://nanohub.org/resources/particleve} \hfill\doi{10.4231/D35D8NF30}
                        \end{innerlist}
                    \item[] NanoFET
                        \begin{innerlist}
                            \item[] Simulates the effects
                              of downscaling conventional CMOS devices,
                              uncovering the challenges of working on the
                              ever-shrinking nanoelectronics in devices.
                              Worked with a small team of Electrical Engineering
                              postdocs to parallelize the code using MPI,
                              configure it to run on TeraGrid supercomputers,
                              and deploy it on \href{https://nanohub.org}{nanoHUB.org}
                              as a publicly available community software
                              tool.
                            \item[] Coded in FORTRAN
                            \item[] \url{https://nanohub.org/resources/nanofet} \hfill\doi{10.4231/D3X921K5T}
                        \end{innerlist}
                \end{outerlist}

            \halfblankline

% 4. open source code
            \item \textbf{Created \href{https://github.com/codedsk/hubcheck}{HUBcheck},
                an open source Python library} used to build automation scripts
                and user level tests for HUBzero based websites and simulation tool
                environments.  Built on top of
                \href{http://www.seleniumhq.org/projects/webdriver}{Selenium WebDriver}
                and \href{https://www.mozilla.org/en-US/firefox/new}{Firefox}
                to provide web browser automation,
                \href{http://www.paramiko.org}{Paramiko} to provide SSH automation,
                \href{https://bmp.lightbody.net}{BrowserMob Proxy},
                \href{https://www.ffmpeg.org}{FFmpeg},
                \href{http://www.karlrunge.com/x11vnc/}{VNC},
                and X server utilities. With HUBcheck, developers can
                simulate a user's website experience through abstractions of HUB
                web pages and interact with the HUB's virtualized, Debian GNU/Linux
                based simulation tool environment, all from a single script.\\
                \url{https://github.com/codedsk/hubcheck}

            \halfblankline

% 1. help drive adoption
% 5. help community
            \item \textbf{Cultivated next generation researchers through mentorship and
                teaching} of students participating in Purdue University's Summer
                Undergraduate Research Fellowship (SURF) program. Organized workshops,
                held open office hours, and worked with other faculty and staff to introduce
                students to nanotechnology research, software development
                best practices, the \HUBZEROPLATFORM, and \RAPPTURE.\\
                \url{http://www.purdue.edu/surf}

        \end{innerlist}


\section{Professional History}

        \textit{Software Engineer},
                \href{http://www.purdue.edu/}{Purdue University} %
                \hfill \textbf{2005 - Present} \\
                Developer for \HUBZEROPLATFORMSCICOL

        \halfblankline

        \textit{Associate Programmer},
                \href{http://www.convergys.com/}{Convergys Corporation} %
                \hfill \textbf{2004 - 2005} \\
                Developer for Mediation Manager, mobile phone billing and
                rating software.


\section{Education}

      Purdue University \\
                \textbf{Master of Science (MS)},
                \href{https://engineering.purdue.edu/ECE}
                     {Electrical and Computer Engineering}
                \hfill \textbf{May 2015} \\
                Thesis Topic:
                  \href{http://docs.lib.purdue.edu/dissertations/AAI1597870/}
                    {\emph{Automated testing in multimodal systems}}

\halfblankline

      Purdue University \\
                \textbf{Bachelor or Science (BS)},
                \href{https://engineering.purdue.edu/ECE}
                     {Computer Engineering}
                \hfill \textbf{Dec 2003} \\
                Emphasis on software systems

\halfblankline

\section{Publications, Proceedings, Talks}

    \begin{lonelist}
        \item \textbf{Co-authored 7 refereed journal publications and conference
          proceedings} with topics ranging from
          \href{http://dx.doi.org/10.1002/cpe.3257}{building web-based
          scientific workflows} to Remote Sensing tools and applications.\\
          \url{http://bit.ly/dk-pub-2014-1} \hfill \doi{10.1002/cpe.3257}
        \item \textbf{Presented at 3 conferences} about
          \href{https://www.youtube.com/watch?v=AVrnBJDQeaI}{design
          patterns for Selenium WebDriver based automated website testing} and
          \href{https://nanohub.org/groups/nanobio/nano_agenda_april_2015}
          {deploying scientific simulation codes on the web}. \\
          \url{https://www.youtube.com/watch?v=AVrnBJDQeaI}
        \item \textbf{Led 5 \href{https://nanohub.org/courses/tools/hubbub2015}
          {interactive workshops}}, with lectures and assignments, teaching
          software best practices and methods for building and deploying
          scientific simulation tools on the \HUBZERO\space Platform. \\
          \url{https://nanohub.org/courses/tools/hubbub2015}
    \end{lonelist}

\halfblankline

\section{Service}

    \href{http://bit.ly/caribsa/}{Purdue Caribbean Student Association},
      Advisor
      \hfill \textbf{2005 - Present} \\
    \href{http://www.meetup.com/Greater-Lafayette-Open-Source-Symposium}{Greater Lafayette Open Source Symposium/PurduePM},
      Presenter
      \hfill \textbf{2007 - Present} \\
    \href{http://www.lthc.net/}{Lafayette Transitional Housing Center},
      Volunteer Chef
      \hfill \textbf{2012 - 2015} \\
    Boy Scout Troop 336, West Lafayette, IN,
      Committee Chair
      \hfill \textbf{2012 - 2013} \\
    \href{http://www.purdue.edu/mep/}{Purdue Minority Engineering Program},
      Summer Programs Volunteer
      \hfill \textbf{2005 - 2008}


\section{Awards}
%
    Eagle Scout


\section{More Information}
%
    Learn more at \textbf{\url{http://bit.ly/dk-cv}}.\\

\end{document}

%%%%%%%%%%%%%%%%%%%%%%%%%% End CV Document %%%%%%%%%%%%%%%%%%%%%%%%%%%%%

%----------------------------------------------------------------------%
% The following is copyright and licensing information for
% redistribution of this LaTeX source code; it also includes a liability
% statement. If this source code is not being redistributed to others,
% it may be omitted. It has no effect on the function of the above code.
%----------------------------------------------------------------------%
% Copyright (c) 2007, 2008, 2009, 2010, 2011 by Theodore P. Pavlic
%
% Unless otherwise expressly stated, this work is licensed under the
% Creative Commons Attribution-Noncommercial 3.0 United States License. To
% view a copy of this license, visit
% http://creativecommons.org/licenses/by-nc/3.0/us/ or send a letter to
% Creative Commons, 171 Second Street, Suite 300, San Francisco,
% California, 94105, USA.
%
% THE SOFTWARE IS PROVIDED "AS IS", WITHOUT WARRANTY OF ANY KIND, EXPRESS
% OR IMPLIED, INCLUDING BUT NOT LIMITED TO THE WARRANTIES OF
% MERCHANTABILITY, FITNESS FOR A PARTICULAR PURPOSE AND NONINFRINGEMENT.
% IN NO EVENT SHALL THE AUTHORS OR COPYRIGHT HOLDERS BE LIABLE FOR ANY
% CLAIM, DAMAGES OR OTHER LIABILITY, WHETHER IN AN ACTION OF CONTRACT,
% TORT OR OTHERWISE, ARISING FROM, OUT OF OR IN CONNECTION WITH THE
% SOFTWARE OR THE USE OR OTHER DEALINGS IN THE SOFTWARE.
%----------------------------------------------------------------------%
